\documentclass{rapportECC}
\usepackage{lipsum}
\title{Rapport ECL - Template} %Titre du fichier
\usepackage{lipsum} 
\usepackage{biblatex} %Imports biblatex package
\addbibresource{bibtex.bib} %Import the bibliography file
\usepackage{appendix} % Package pour gérer les annexes
\begin{document}

%----------- Informations du rapport ---------

\titre{Compte rendu TER semestre 5} %Titre du fichier .pdf

\sujet{Sujet 9 : Étude de l’élévation thermique dans un multicouche exposé à un laser à impulsions } %Nom du sujet

\Encadrants{Zouhir \textsc{Maslah}
 \\Gaël \textsc{Poette} } %Nom de l'enseignant

\eleves{Matéo \textsc{Ringeval} \\
		Soren \textsc{Le Meitour} \\
            Adrien \textsc{Mayeux - Morlans} \\
		Mircea \textsc{Cathelin} } %Nom des élèves

%----------- Initialisation -------------------
        
\fairemarges %Afficher les marges
\fairepagedegarde %Créer la page de garde
\tabledematieres %Créer la table de matières

%------------ Corps du rapport ----------------


\section{Présentation du sujet} 

\subsection{Introduction}

Le but de ce projet est d'étudier l'élévation thermique dans un multicouche soumis à des impulsions laser courtes (de l'ordre de la femtoseconde) à l'aide d'un second laser. L'avantage de ce genre de méthodes est qu'elles permettent par exemple de mesurer de manière précise la température de micro-composants électroniques. 
Afin de réaliser cette étude, nous nous baserons sur les données obtenues durant l'expérience décrite en partie 3 et réaliserons un modèle sous fortran90. Pour évaluer la pertinence de notre modèle nous ferons des évaluations de fréquence Brillouin.

\subsection{Démarche}

Nous avons choisis comme démarche de commencer par faire des recherches bibliographiques afin d'obtenir une équation pour la fréquence Brillouin ainsi que des valeurs des différentes variables en fonction de la température. Puis, nous avons réalisé l'expérience afin de récolter des données qui nous seront utiles dans la suite pour vérifier notre modèle. Vient ensuite la construction du modèle à partir de l'équation différentielle de la chaleur. Pour ce faire, nous avons commencé par la résoudre pour notre système. Nous prévoyons pour la suite de finir le modèle.

\section{Premiere phase de recherche bibliographique}

Nous avons tout d’abord fait des recherches bibliographique pour trouver les formules suivantes :

\begin{itemize}
    \item v  la vitesse du son dans l’eau en fonction de la température T
    \item n l’indice de réfraction dans l’eau en fonction de la température T
    \item f la fréquence Brillouin en fonction de v, n et T
\end{itemize}

Toutes ces informations proviennent ainsi des articles suivants trouvés sur internet :
\cite{}


\subsection{Formules obtenues}

Nous avons donc pu en tirer les formules suivantes :

Pour la vitesse du son dans un liquide
\begin{equation}
    v = \sqrt{\frac{1}{\rho \beta_{ad}}}
\end{equation}
où $\rho$ est la masse vlumique de l'eau et $\beta_{ad}$ est compressibilité adiabatique

Pour l'indice de réfraction 
\begin{equation}
    -\frac{\partial n}{\partial t} = B(1-exp(-\frac{t-t_0}{t_k}))
\end{equation}
avec $B=26,2*10^{-5} K^{-1}, t_0$ = 20 °C et h = 48,5 °C

Enfin la formule de la fréquence Brillouin
\begin{equation}
    f_B = \frac{2vn}{\lambda}\sqrt{1-\frac{\sin^2{\theta}}{n^2}}
\end{equation}

Mais dans notre cas, le laser est perpendiculaire à la source donc nous pouvons simplifier notre formule de $f_B$ pour obtenir :
\begin{equation}
    f_B =  \frac{2vn}{\lambda}
\end{equation}

\section{Expérience}

Le principe de l'expérience est d'envoyer un laser source à impulsions sur le multicouche (saphir, titane, eau) en passant par le saphir ce qui provoque une élévation de température et donc une dilatations thermique associée à une onde mécanique mesurées à l'aide de deux lasers source. Nous avons ainsi récupéré des valeurs d'intensité lumineuse (réflexion des deux laser sonde) sur une grille de 6x6 pour 5k intervalles de temps.

\section{Recherche de modèle}

\subsection{Expression de T}

En partant de la formule de 

Comme nous cherchons à déterminer l’élévation de la température statique,
nous utiliserons la transformée de Fourier pour passer dans le domaine
fréquentiel. Nous utiliserons également la transformée de Hankel pour convertir
l’équation aux dérivées partielles en une équation aux dérivées ordinaires.
L’équation (1) devient alors :

\begin{equation}
    \frac{\partial^2 T(k,z,\omega)}{\partial z^2} - (k^2 + j\frac{\omega}{\alpha}) T(k,z,\omega) = - \frac{1}{\kappa_1} Q_i(k,z,\omega)
\end{equation}

Cependant donc notre cas nous pouvons faire des hypothéses simplificatrices.

Ainsi, nous pouvons négliger la partiel imaginaire de l'équation donc négliger la dépendance en $\omega$ et z de l'équation.

Cela nous permet d'avoir une expression de T comme celle-ci :

\begin{equation}
    \frac{\partial^2 T(z)}{\partial z^2} - k^2 T = \frac{1}{\kappa_1}Q_i(z)
\end{equation}
Avec $Q_i$ = $\beta_i e^{-\beta_i z}$


Ainsi nous nous retrouvons à résoudre une équation différentiel du second ordre.\\
Nous avons choisis de la résoudre en utilisant variation de la constante.

Comme notre problème comporte 3 milieux différentes,
\begin{itemize}
    \item Le milieu (0) étant celui du saphir
    \item Le milieu (1) étant celui du Titane
    \item Le milieu (2) étant celui de l'eau
\end{itemize}

Le laser ne provoque pas d'échauffement dans les milieux (0) et (2) du fait de leur transparence, le terme source $Q_i$ est donc nul, ainsi, dans ces deux milieux l'équation devient :

\begin{equation}
    \frac{\partial^2 T(z)}{\partial z^2} - k^2 T = 0
\end{equation}

Ainsi, nous nous retrouvons à simplifier encore plus l'équation dans le cas (0) et (2) et n'avoir seulement besoin de $T_h$

\begin{equation*}
    \frac{d^2 T(z)}{d z^2} - k^2 T(z) = - \frac{1}{\kappa_1}\beta_i e^{-\beta_iz}
\end{equation*}

d'où,
\begin{equation}
    T_h = A e^{kz} 
\end{equation}
avec A $\in \mathbb{R}$

Enfin pour calculer $T_P$ on utilise la variation de la constante ce qui nous donne :

\begin{equation*}
            T_P = A''(z)e^{(kz)} + 2k A'(z)e^{(kz)} = - \frac{1}{\kappa_1}\beta_i e^{-\beta_iz} 
\end{equation*}

\begin{equation*}
            => - \frac{1}{\kappa_1}\beta_i e^{(-\beta_i + k)z} = A''(z) + 2k A'(z)
\end{equation*}

On effectue une nouvelle fois la variation de la constante :

\begin{equation*}
    A'_h = B e^{-2kz}
\end{equation*}
\begin{equation*}
    A'_p = B(z) e^{-2kz}
\end{equation*}
ce qui donne
\begin{equation*}
   B'(z)e^{-2kz} = -\frac{1}{\kappa_1}\beta_1 e^{-(\beta_1+k)z}\\
   \xrightarrow{} B'(z) = -\frac{1}{\kappa_1}\beta_1 e^{(k-\beta_1)z}
   \xrightarrow{} B(z) = \frac{\beta_1}{\kappa_1(k-\beta_1)}e^{(k-\beta_1)z} + C
\end{equation*}
On a donc l'expression de $A'(z)$
\begin{equation*}
   A'(z)=(B+C)e^{-2kz} -\frac{-\beta_1}{\kappa_1(k-\beta_1)} e^{-(\beta_1+k)z}\\
   \xrightarrow{} A(z) = -\frac{B+C}{2k}e^{-2kz}+\frac{\beta_1}{\kappa_1(k^2-\beta_1^2)} e^{-(\beta_1+k)z}
\end{equation*}
On obtient ainsi $T(z) = T_p + T_h$
\begin{equation*}
            T(z) = \left[A+\frac{\beta_1}{\kappa_1(k^2-\beta_1^2)}e^{-(\beta_1+k)z}-\frac{B+C}{2k}e^{-2kz}+D\right]e^{kz} 
\end{equation*}






















% %------------- Commandes utiles ----------------



% Voici quelques commandes utiles : \cite{Lamport}

% %------ Pour insérer et citer une image centralisée -----

% \insererfigure{logos/Logo_ECC}{3cm}{Légende de la figure}{Label de la figure}
% % Le premier argument est le chemin pour la photo
% % Le deuxième est la hauteur de la photo
% % Le troisième la légende
% % Le quatrième le label
% Ici, je cite l'image \ref{fig: Label de la figure}


% %------- Pour insérer et citer une équation --------------

% \begin{equation} \label{eq: exemple}
% \rho + \Delta = 42
% \end{equation}

% L'équation \ref{eq: exemple} est cité ici. 

% % ------- Pour écrire des variables ----------------------

% Pour écrire des variables dans le texte, il suffit de mettre le symbole \$ entre le texte souhaité comme : constante $\rho$. \cite{Companion}


\section{Conclusion et Perspectives}


\newpage


\printbibliography

% Ajout de l'annexe
\newpage
\begin{appendices}
\section{Annexe : }
Dans cette section de l'annexe, nous fournissons des détails supplémentaires.

% Ajout d'un tableau
\subsection{Tableau additionnel}
Voici un exemple de tableau :
\begin{table}[h]
    \centering
    \begin{tabular}{|c|c|}
    \hline
    Colonne 1 & Colonne 2 \\
    \hline
    Donnée A & 1 \\
    Donnée B & 2 \\
    \hline
    \end{tabular}
    \caption{Exemple de tableau}
\end{table}

% Ajout d'une image
\subsection{Graphique complémentaire}
Voici un exemple d'insertion d'une image :
\begin{figure}[h]
    \centering
    \includegraphics[width=0.5\textwidth]{logos/centrale.png}
    \caption{Légende de l'image}
\end{figure}
\end{appendices}

\end{document}
